\section{fmm.c File Reference}
\label{fmm_8c}\index{fmm.c@{fmm.c}}
fast marching method 



\subsection{Detailed Description}
fast marching method 

{\bf Usage:} fmm speed.pgm seeds.pgm stop threshold seedout.pgm distanceout.pgm

{\bf Description:}

The fast marching method of J.A. Sethian is a geodesic distance transform. It integrates the constant arrival hyperbolic PDE. A rough physical analogy is the following: Assume an anisotropic medium with varying propagation celerities, and waves starting from various seeds travelling through that medium. This function computes the successive arrival times of these waves, as well as the propagation of the initial labels. The successive arrival times are equivalent to a distance transform. The propagation of the label yield a partition similar to a Voronoi.

If the speed function is constant and equal to 1, the arrival times would indeed be the Euclidean distance function and the partition the Euclidean Voronoi. Due to discretisation issue, the result is only approximately Euclidiean (to second order).

{\bf Stopping criteria:} stop is the stopping criteria :

\begin{itemize}
\item stop = 0 =$>$ no stop \item stop = 1 =$>$ stop on metric (if speed function $>$= threshold) \item stop = 2 =$>$ stop on distance (if distance $>$ threshold)\end{itemize}
the threshold is given after.

{\bf Types supported:} integer, float Nd (N $>$= 2) speed must be float, seeds must be integer.

{\bf Category:} morpho

\begin{Desc}
\item[Author:]Hugues Talbot and Ben Appleton \end{Desc}
