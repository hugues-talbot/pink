\section{longestplateau.c File Reference}
\label{longestplateau_8c}\index{longestplateau.c@{longestplateau.c}}


finds the location of the longest plateau in 1D sequence  




\subsection{Detailed Description}
finds the location of the longest plateau in 1D sequence {\bfseries Usage:} longestplateau in.list [tolerance] out.list

{\bfseries Description:}

Reads the sequence S from the file {\bfseries in.list}. This file must have the following format: 
\begin{DoxyPre}  
  e <n>
  x1
  x2
  ...
  xn
\end{DoxyPre}
 The tolerance t (float) is given by the parameter {\bfseries tolerance} (default value is 0). A {\itshape plateau\/} is a subsequence P of S formed by consecutive elements of P, between indices i and j, and such that max\{P[k];i$<$=k$<$=j\} -\/ min\{P[k];i$<$=k$<$=j\} $<$= t. The program returns the base index and length of the first occurence of a plateau with maximal length in S.

{\bfseries Types supported:} list 1D

{\bfseries Category:} signal

\begin{DoxyAuthor}{Author}
Michel Couprie 
\end{DoxyAuthor}
