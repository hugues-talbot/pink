\section{longestplateau.c File Reference}
\label{longestplateau_8c}\index{longestplateau.c@{longestplateau.c}}
finds the location of the longest plateau in 1D sequence 



\subsection{Detailed Description}
finds the location of the longest plateau in 1D sequence 

{\bf Usage:} longestplateau in.list [tolerance] out.list

{\bf Description:}

Reads the sequence S from the file {\bf in.list}. This file must have the following format: \small\begin{alltt}  
  e <n>
  x1
  x2
  ...
  xn
\end{alltt}
\normalsize 
 The tolerance t (float) is given by the parameter {\bf tolerance} (default value is 0). A {\em plateau\/} is a subsequence P of S formed by consecutive elements of P, between indices i and j, and such that max\{P[k];i$<$=k$<$=j\} - min\{P[k];i$<$=k$<$=j\} $<$= t. The program returns the base index and length of the first occurence of a plateau with maximal length in S.

{\bf Types supported:} list 1D

{\bf Category:} signal

\begin{Desc}
\item[Author:]Michel Couprie \end{Desc}
