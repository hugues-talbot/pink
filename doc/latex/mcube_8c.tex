\section{mcube.c File Reference}
\label{mcube_8c}\index{mcube.c@{mcube.c}}
topologically correct \char`\"{}marching cubes\char`\"{}-like algorithm 



\subsection{Detailed Description}
topologically correct \char`\"{}marching cubes\char`\"{}-like algorithm 

{\bf Usage:} mcube in.pgm threshold nregul obj\_\-id format [connex] out

{\bf Description:} Generates a 3d mesh from the binary or grayscale image {\bf in.pgm} .

The original image is first thresholded (parameter {\bf threshold}, values 0 and 1 both fit for a binary image). Then the method described in [Lac96] is applied to generate the 3d mesh. The resulting mesh is then smoothed by {\bf nregul} steps of laplacian smoothing. The parameter {\bf obj\_\-id} is used to tag the generated mesh.

The parameter {\bf format} indicate the format of the output file (choices are POV, POVB, COL, MCM, AC, GL, VTK, RAW). The keyword POVB corresponds to a bare Povray mesh: a header and a footer must be catenated in order to make a full Povray scene. The keyword RAW is the exchange format for the \char`\"{}mesh\char`\"{} software: see {\tt http://mesh.berlios.de/}

The optional parameter {\bf connex} indicates the connexity used for the object. Possible values are 6 and 26 (default).

[Lac96] J.-O. Lachaud, \char`\"{}Topologically defined iso-surfaces\char`\"{}, DGCI'96, LNCS 1176 (245--256), Springer Verlag, 1996.

{\bf Types supported:} byte 3d

{\bf Category:} mesh3d

\begin{Desc}
\item[Author:]Michel Couprie \end{Desc}
