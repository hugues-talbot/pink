\section{ultimateerosion.c File Reference}
\label{ultimateerosion_8c}\index{ultimateerosion.c@{ultimateerosion.c}}
ultimate erosion 



\subsection{Detailed Description}
ultimate erosion 

{\bf Usage:} ultimateerosion in.pgm [dist] out.pgm

{\bf Description:} Let X be the set in {\bf in.pgm} . The result is union\{Ui(X), i in N\} where Ui(X) = erosball(X,i) $\backslash$ reconsgeo(erosball(X,i+1), erosball(X,i)). Structuring elements are balls defined after a distance. The distance used depends on the optional parameter {\bf dist} (default is 0) : \begin{itemize}
\item 0: approximate euclidean distance (truncated) \item 1: approximate quadratic euclidean distance \item 2: chamfer distance \item 3: exact quadratic euclidean distance \item 4: 4-distance in 2d \item 8: 8-distance in 2d \item 6: 6-distance in 3d \item 18: 18-distance in 3d \item 26: 26-distance in 3d\end{itemize}
\begin{Desc}
\item[Warning:]The input image {\bf in.pgm} must be a binary image. No test is done.\end{Desc}
{\bf Types supported:} byte 2D, byte 3D

{\bf Category:} morpho

\begin{Desc}
\item[Author:]Michel Couprie ao\~{A}�t 2009 \end{Desc}
