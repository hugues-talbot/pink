\section{ultimateerosion.c File Reference}
\label{ultimateerosion_8c}\index{ultimateerosion.c@{ultimateerosion.c}}


ultimate erosion  




\subsection{Detailed Description}
ultimate erosion {\bfseries Usage:} ultimateerosion in.pgm [dist] out.pgm

{\bfseries Description:} Let X be the set in {\bfseries in.pgm} . The result is union\{Ui(X), i in N\} where Ui(X) = erosball(X,i) $\backslash$ reconsgeo(erosball(X,i+1), erosball(X,i)). Structuring elements are balls defined after a distance. The distance used depends on the optional parameter {\bfseries dist} (default is 0) : \begin{DoxyItemize}
\item 0: approximate euclidean distance (truncated) \item 1: approximate quadratic euclidean distance \item 2: chamfer distance \item 3: exact quadratic euclidean distance \item 4: 4-\/distance in 2d \item 8: 8-\/distance in 2d \item 6: 6-\/distance in 3d \item 18: 18-\/distance in 3d \item 26: 26-\/distance in 3d\end{DoxyItemize}
\begin{DoxyWarning}{Warning}
The input image {\bfseries in.pgm} must be a binary image. No test is done.
\end{DoxyWarning}
{\bfseries Types supported:} byte 2D, byte 3D

{\bfseries Category:} morpho

\begin{DoxyAuthor}{Author}
Michel Couprie août 2009 
\end{DoxyAuthor}
