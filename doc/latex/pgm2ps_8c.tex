\section{pgm2ps.c File Reference}
\label{pgm2ps_8c}\index{pgm2ps.c@{pgm2ps.c}}
converts from pgm to ps for illustrating small image configurations  




\label{_details}
\subsection{Detailed Description}
converts from pgm to ps for illustrating small image configurations 

{\bf Usage:} pgm2ps in.pgm mode $<$label.pgm$|$null$>$ $<$marker.pgm$|$null$>$ coord maxval out.ps

{\bf Description:} Produces a Postscript file from a binary or grayscale image. If {\bf mode} = \begin{itemize}
\item b: binary image \item c: grayscale image (levels not printed) \item n: grayscale image (levels printed as numbers on a colored background) \item m: grayscale image (levels printed as numbers on a colored background) \item d: grayscale image (idem n - levels are square-rooted) \item i: grayscale image (levels printed as numbers on an inverted colored background) \item p: grayscale image (levels printed as numbers) \item a: grayscale image (levels printed as letters: a=1, b=2...) \item g: grayscale image (levels showed as colored items) \item v: vector image \item B: binary khalimsky grid \item N: grayscale khalimsky grid (levels printed as numbers) \item G: grayscale khalimsky grid (levels showed as colored items) \item C: binary khalimsky grid (dual of B) \item M: grayscale khalimsky grid (dual of N) \item H: grayscale khalimsky grid (dual of G)\end{itemize}
If an image {\bf label.pgm} is present and the mode is n, the level number will be replaced by a letter for each pixel not null in the label image.

If an image {\bf label.pgm} is present and the mode is m, the printed number will taken in the image {\bf label.pgm} while the color will be taken from {\bf in.pgm} .

If an image {\bf marker.pgm} is present, a circle will be drawn on each point not null in the marker image.

The parameter {\bf coord} is a int32\_\-t (0 or 1) which commands the inclusion of axis coordinates in the figure.

The parameter {\bf maxval} is an integer which indicates the maximum grayscale value of the figure (not necessarily the maximum value of the input image).

{\bf Types supported:} byte 2d, int32\_\-t 2d

{\bf Category:} convert

\begin{Desc}
\item[Author:]Michel Couprie \end{Desc}
