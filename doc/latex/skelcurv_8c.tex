\section{skelcurv.c File Reference}
\label{skelcurv_8c}\index{skelcurv.c@{skelcurv.c}}
curvilinear binary skeleton guided by a priority image 



\subsection{Detailed Description}
curvilinear binary skeleton guided by a priority image 

{\bf Usage:} skelcurv in.pgm prio connex [inhibit] out.pgm

{\bf Description:} Curvilinear binary skeleton guided by a priority image. The lowest values of the priority image correspond to the highest priority.

The parameter {\bf prio} is either an image (byte or int32\_\-t or float or double), or a numerical code indicating that a distance map will be used as a priority image; the possible choices are: \begin{itemize}
\item 0: approximate euclidean distance \item 1: approximate quadratic euclidean distance \item 2: chamfer distance \item 3: exact quadratic euclidean distance \item 4: 4-distance in 2d \item 8: 8-distance in 2d \item 6: 6-distance in 3d \item 18: 18-distance in 3d \item 26: 26-distance in 3d\end{itemize}
The parameter {\bf connex} indicates the connectivity of the binary object. Possible choices are 4, 8 in 2d and 6, 26 in 3d.

If the parameter {\bf inhibit} is given and is a binary image name, then the points of this image will be left unchanged.

Let X be the set corresponding to the input image {\bf in.pgm}. Let P be the function corresponding to the priority image. Let I be the set corresponding to the inhibit image, if given, or the empty set otherwise. The algorithm is the following:



\footnotesize\begin{verbatim}C = null image
Repeat until stability
  choose a point x in X \ I, simple for X, such that C[x] == 0 
    and such that P[x] is minimal
  X = X \ {x}
  for all y in gamma(x)
    if T(y) > 1 then C[y] = 1
Result: X
\end{verbatim}
\normalsize


where T(y) refers to the connectivity number of y, that is, (informally) the number of connected components of gamma(y) inter (X $\backslash$ \{y\}).

References:\par
 [BC07] G. Bertrand and M. Couprie: {\tt \char`\"{}Transformations topologiques discretes\char`\"{}}, in {\em G\'{e}om\'{e}trie discr\`{e}te et images num\'{e}riques\/}, D. Coeurjolly and A. Montanvert and J.M. Chassery, pp.~187-209, Herm\`{e}s, 2007.\par


{\bf Types supported:} byte 2d, byte 3d

{\bf Category:} topobin

\begin{Desc}
\item[Author:]Michel Couprie \end{Desc}
