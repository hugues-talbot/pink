\section{rgb2hls.c File Reference}
\label{rgb2hls_8c}\index{rgb2hls.c@{rgb2hls.c}}
converts a ppm color file to hls representation 



\subsection{Detailed Description}
converts a ppm color file to hls representation 

{\bf Usage:} in.ppm [mode] h.pgm l.pgm s.pgm

{\bf Description:} Constructs 3 pgm files from 1 ppm file: \begin{itemize}
\item {\bf h.pgm} : Hue (int32\_\-t - 0 to 359) \item {\bf h.pgm} : Luminance (byte) \item {\bf h.pgm} : Saturation (byte)\end{itemize}
Different modes are available [default mode is 0]: \begin{itemize}
\item mode = 0: classical HLS coding, based on [1]. \item mode = 1: L1 norm (NYI) (see [2]) \item mode = 2: L1 norm with gamma correction (NYI) (see [2])\end{itemize}
[1] Foley, Van Damm \& al: \char`\"{}Computer Graphics\char`\"{}, 2nd ed., p. 595

[2] J. Angulo and J. Serra. \char`\"{}Traitements des images de couleur en repr\'{e}sentation luminance/saturation/teinte par norme L\_\-1\char`\"{} (in French). Traitement du Signal, Vol. 21, No. 6, pp 583-604, December 2004.

{\bf Types supported:} byte 2d

{\bf Category:} convert

\begin{Desc}
\item[Author:]Michel Couprie \end{Desc}
