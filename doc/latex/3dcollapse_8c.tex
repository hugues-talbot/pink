\section{3dcollapse.c File Reference}
\label{3dcollapse_8c}\index{3dcollapse.c@{3dcollapse.c}}
ultimate constrained collapse guided by a priority image 



\subsection{Detailed Description}
ultimate constrained collapse guided by a priority image 

{\bf Usage:} 3dcollapse in.pgm prio [inhibit] out.pgm

{\bf Description:} Ultimate constrained collapse guided by a priority image. The lowest values of the priority image correspond to the highest priority.

The parameter {\bf prio} is either an image (byte or int32\_\-t), or a numerical code indicating that a distance map will be used as a priority image; the possible choices are: \begin{itemize}
\item 0: approximate euclidean distance \item 1: approximate quadratic euclidean distance \item 2: chamfer distance \item 3: exact quadratic euclidean distance \item 6: 6-distance in 3d \item 18: 18-distance in 3d \item 26: 26-distance in 3d\end{itemize}
If the parameter {\bf inhibit} is given and is a binary image name, then the elements of this image will be left unchanged. If the parameter {\bf inhibit} is given and is a number I, then the elements with priority greater than or equal to I will be left unchanged.

\begin{Desc}
\item[Warning:]The result makes sense only if the input image is a complex.\end{Desc}
{\bf Types supported:} byte 3d

{\bf Category:} orders

\begin{Desc}
\item[Author:]Michel Couprie \end{Desc}
