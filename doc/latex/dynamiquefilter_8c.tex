\section{dynamiquefilter.c File Reference}
\label{dynamiquefilter_8c}\index{dynamiquefilter.c@{dynamiquefilter.c}}
filter components according to the dynamics of the maxima  




\label{_details}
\subsection{Detailed Description}
filter components according to the dynamics of the maxima 

{\bf Usage:} dynamiquefilter in.pgm connex value [order] out.pgm

{\bf Description:} Computes the ordered dynamics of the maxima, with connectivity {\bf connex}, selects the maxima with a dynamics greater or equal to {\bf value}, and find the maximal components which includes these maxima. The definition of the ordered dynamics is the one given in [Ber05]. The optional argument {\bf order} is one of the following: \begin{itemize}
\item 0: altitude [default] \item 1: area \item 2: volume\end{itemize}
References:\par
 [Ber05] G. Bertrand: \char`\"{}A new definition of the dynamics\char`\"{}, {\em Procs. ISMM05\/}, Springer, series Computational Imaging and Vision, Vol.~30, pp.~197-206, 2005.\par


{\bf Types supported:} byte 2D, byte 3D.

{\bf Category:}

\begin{Desc}
\item[Author:]Michel Couprie \end{Desc}
