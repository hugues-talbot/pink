\section{pgm2curve.c File Reference}
\label{pgm2curve_8c}\index{pgm2curve.c@{pgm2curve.c}}


extracts a curve from a binary image  




\subsection{Detailed Description}
extracts a curve from a binary image {\bfseries Usage:} pgm2curve image.pgm connex [x y [z]] out.curve

{\bfseries Description:} Extracts a curve from a binary image. The parameter {\bfseries connex} is the connexity of the curve. It may be equal to 4 or 8 in 2D, and to 6, 18 or 26 in 3D. If given, the point {\bfseries (x, y, z)} (2D) or {\bfseries (x, y, z)} (3D) is the beginning of the curve, and must be an end point. The output is the text file {\bfseries out.curve}, with the following format:\par
 c nbpoints\par
 x1 y1\par
 x2 y2\par
 ...\par
 or (3D): C nbpoints\par
 x1 y1 z1\par
 x2 y2 z2\par
 ...\par


The points of the curve may also be valued. This is must be indicated by a value of 40, 80, 60, 180 or 260 for the parameter {\bfseries connex}, instead of 4, 8, 6, 18 or 26 respectively. In this case, the output is the text file {\bfseries out.curve}, with the following format:\par
 cv nbpoints\par
 x1 y1 v1\par
 x2 y2 v2\par
 ...\par
 or (3D): CV nbpoints\par
 x1 y1 z1 v1\par
 x2 y2 z2 v2\par
 ...\par


{\bfseries Types supported:} byte 2D, byte 3D

{\bfseries Category:} convert geo

\begin{DoxyAuthor}{Author}
Michel Couprie 
\end{DoxyAuthor}
