\section{hthin.c File Reference}
\label{hthin_8c}\index{hthin.c@{hthin.c}}


grayscale homotopic thinning  




\subsection{Detailed Description}
grayscale homotopic thinning {\bfseries Usage:} hthin in.pgm $<$imcond.pgm$|$null$>$ connex niter out.pgm

{\bfseries Description:} Grayscale homotopic thinning (refs. [BEC97, CBB01]). The parameter {\bfseries connex} gives the connectivity used for the minima; possible choices are 4, 8 in 2D and 6, 26 in 3D. Let F be the function corresponding to the input image {\bfseries in.pgm}. Let G be the function corresponding to the input image {\bfseries imcond.pgm}, or the null function if the keyword {\bfseries null} is used. The algorithm is the following:

\begin{DoxyVerb}
Repeat niter times:
    L = {(p,d) such that p is destructible for F and d = delta-(p,F)}
    While L not empty
       extract a couple (p,d) from L
       F(p) = max{ d, delta-(p,F), G(p) }    
Result: F
\end{DoxyVerb}


If {\bfseries niter} is set to -\/1, then the operator iterates until stability.

References:

[BEC97] G. Bertrand, J. C. Everat and M. Couprie: {\tt \char`\"{}Image segmentation through operators based upon topology\char`\"{}}, {\itshape  Journal of Electronic Imaging\/}, Vol.~6, No.~4, pp.~395-\/405, 1997.

[CBB01] M. Couprie, F.N. Bezerra, Gilles Bertrand: {\tt \char`\"{}Topological operators for
grayscale image processing\char`\"{}}, {\itshape  Journal of Electronic Imaging\/}, Vol.~10, No.~4, pp.~1003-\/1015, 2001.

{\bfseries Types supported:} byte 2D, byte 3D.

{\bfseries Category:} topogray

\begin{DoxyAuthor}{Author}
Michel Couprie
\end{DoxyAuthor}
{\bfseries Example:}

hthin ur1 null 4 10 ur1\_\-hthin\par
 minima ur1\_\-hthin 4 ur1\_\-hthin\_\-m

\begin{TabularC}{3}
\hline
 & &  \\\cline{1-3}
ur1 &ur1\_\-hthin &ur1\_\-hthin\_\-m  \\\cline{1-3}
\end{TabularC}
