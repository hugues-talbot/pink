\section{diZenzo.c File Reference}
\label{diZenzo_8c}\index{diZenzo.c@{diZenzo.c}}
diZenzo gradient pour les images couleurs  




\label{_details}
\subsection{Detailed Description}
diZenzo gradient pour les images couleurs 

{\bf Usage:} diZenzo imageRVB.ppm alpha [mode] out.pgm

{\bf Description:} Le gradient de diZenzo est d�fini par la donn�e de p, q, et t:

p = Rx$\ast$Rx + Vx$\ast$Vx + Bx$\ast$Bx

q = Ry$\ast$Ry + Vy$\ast$Vy + By$\ast$By

t = Rx$\ast$Ry + Vx$\ast$Vy + Bx$\ast$By

(ou Rx = d�riv�e en x de la bande rouge, Ry est la d�riv�e en y de la bande rouge, etc.)

et le module est donn�e par

G = sqrt(1/2$\ast$(p+q+sqrt((p+q)$\ast$(p+q) -4(pq-t$\ast$t))))

La direction est donn�e par 1/2$\ast$atan(2$\ast$t/(p-q))

Si le mode est �gale � 0 (valeur d�faut) alors l'image de sortie est le gradient, sinon l'image de sortie est une int32\_\-t entre 0 et 360.

Les gradients sont calcul�s par les filtres de Deriche, de param�tre alpha

{\bf Types supported:} byte 2D

{\bf Category:} signal

\begin{Desc}
\item[Author:]Laurent Najman \end{Desc}
