\section{hdilatball.c File Reference}
\label{hdilatball_8c}\index{hdilatball.c@{hdilatball.c}}


topologically controlled dilation  




\subsection{Detailed Description}
topologically controlled dilation {\bfseries Usage:} hdilatball in.pgm radius dist connex out.pgm

{\bfseries Description:} Performs a topologically controlled dilation, that is, a homotopic thickening constrained by the dilation of the input object.

The parameter {\bfseries radius} gives the radius of the ball which is used as structuring element for the dilation.

The parameter {\bfseries dist} is a numerical code indicating the distance which is used to compute the dilation. The possible choices are: \begin{DoxyItemize}
\item 0: approximate euclidean distance \item 1: approximate quadratic euclidean distance \item 2: chamfer distance \item 3: exact quadratic euclidean distance \item 4: 4-\/distance in 2d \item 8: 8-\/distance in 2d \item 6: 6-\/distance in 3d \item 18: 18-\/distance in 3d \item 26: 26-\/distance in 3d\end{DoxyItemize}
The parameter {\bfseries connex} indicates the connectivity of the binary object. Possible choices are 4, 8 in 2d and 6, 26 in 3d.

Let X be the set corresponding to the input image {\bfseries in.pgm}, and let Xbar denote its complementary set. The algorithm is the following:

\begin{DoxyVerb}
I = dilatball(X, dist, r)
Repeat:
    Select a point x in [Xbar inter I] such that dist(x,X) is minimal
    If x is simple for X then
        X = X union {x}
	I = I \ {x}
Until [Xbar inter I] = emptyset
Result: X
\end{DoxyVerb}


{\bfseries Types supported:} byte 2d, byte 3d

{\bfseries Category:} topobin

\begin{DoxyAuthor}{Author}
Michel Couprie 
\end{DoxyAuthor}
