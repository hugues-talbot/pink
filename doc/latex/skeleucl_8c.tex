\section{skeleucl.c File Reference}
\label{skeleucl_8c}\index{skeleucl.c@{skeleucl.c}}
Euclidean binary skeleton.  




\label{_details}
\subsection{Detailed Description}
Euclidean binary skeleton. 

{\bf Usage:} skeleucl in.pgm connex [inhibit] out.pgm

{\bf Description:} Euclidean binary skeleton.

The parameter {\bf connex} indicates the connectivity of the binary object. Possible choices are 4, 8 in 2d and 6, 26 in 3d.

If the parameter {\bf inhibit} is given and is a binary image name, then the points of this image (constraint set) will be left unchanged.

This operator is usually used with a {\em constraint set\/} (parameter {\bf inhibit}) that is a subset of the exact Euclidean medial axis (see operator {\em medialaxis\/}).

Here is an example using the whole medial axis as constraint set:



\begin{footnotesize}\begin{verbatim}
medialaxis test.pgm 3 _1
threshold _1 1 _2
skeleucl test.pgm 8 _2 result.pgm
\end{verbatim}
\end{footnotesize}


Intersesting subsets are obtained by filtering the medial axis, either based on the ball radiuses, or based on the bisector angle (see operator {\em bisector\/}). Below is a script showing how to proceed:



\begin{footnotesize}\begin{verbatim}
#!/bin/sh
USAGE="Usage: $0 in seuilR (in [1..infnty[) seuilA (in [0.001..pi]) out"
if [ $# -ne 4 ]
then
	echo $USAGE
        exit
fi
medialaxis $1 3 /tmp/skel2_tmp_m
threshold /tmp/skel2_tmp_m 1 /tmp/skel2_tmp_m1
skeleucl $1 8 /tmp/skel2_tmp_m1 /tmp/skel2_tmp_s
threshold /tmp/skel2_tmp_m $2 /tmp/skel2_tmp_ms
distc $1 3 /tmp/skel2_tmp_d
bisector /tmp/skel2_tmp_d /tmp/skel2_tmp_ms /tmp/skel2_tmp_a
threshold /tmp/skel2_tmp_a $3 /tmp/skel2_tmp_i
skeleton /tmp/skel2_tmp_s /tmp/skel2_tmp_d 8 /tmp/skel2_tmp_i $4
rm -f /tmp/skel2_tmp_*
\end{verbatim}
\end{footnotesize}


References:\par
 [CCZ07] M. Couprie, D. Coeurjolly and R. Zrour: {\tt \char`\"{}Discrete bisector function and Euclidean skeleton in 2D and 3D\char`\"{}}, {\em Image and Vision Computing\/}, Vol.~25, No.~10, pp.~1543-1556, 2007.\par


{\bf Types supported:} byte 2d, byte 3d

{\bf Category:} topobin

\begin{Desc}
\item[Author:]Michel Couprie \end{Desc}
