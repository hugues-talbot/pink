\section{convol.c File Reference}
\label{convol_8c}\index{convol.c@{convol.c}}
convolution  




\label{_details}
\subsection{Detailed Description}
convolution 

{\bf Usage:} convol in.pgm kernel.pgm [mode] out.pgm

{\bf Description:} Convolution of {\bf in.pgm} by {\bf kernel.pgm}. The result is a float image. Depending on the value given for the (optional) parameter {\bf mode}: \begin{itemize}
\item {\bf mode} = 0 (default) : naive algorithm. The image {\bf in.pgm} is considered as null out of its support. \item {\bf mode} = 1 : naive algorithm. The boundary of image {\bf in.pgm} is extended outside its support. \item {\bf mode} = 2 : convolution using the FFT. The image {\bf in.pgm} is considered as null out of its support. \item {\bf mode} = 3 : convolution using the FFT. The boundary of image {\bf in.pgm} is extended outside its support.\end{itemize}
{\bf Types supported:} byte 2d, int32\_\-t 2d, float 2d

{\bf Category:} signal

\begin{Desc}
\item[Warning:]Naive convolution algorithm is in O($|$in$|$$\ast$$|$kernel$|$). For large kernels, use FFT version which is in O(n log n) where n = max($|$in$|$,$|$kernel$|$)\end{Desc}
\begin{Desc}
\item[Author:]Michel Couprie \end{Desc}
