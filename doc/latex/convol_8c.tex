\section{convol.c File Reference}
\label{convol_8c}\index{convol.c@{convol.c}}


convolution  




\subsection{Detailed Description}
convolution {\bfseries Usage:} convol in.pgm kernel.pgm [mode] out.pgm

{\bfseries Description:} Convolution of {\bfseries in.pgm} by {\bfseries kernel.pgm}. The result is a float image. Depending on the value given for the (optional) parameter {\bfseries mode}: \begin{DoxyItemize}
\item {\bfseries mode} = 0 (default) : naive algorithm. The image {\bfseries in.pgm} is considered as null out of its support. \item {\bfseries mode} = 1 : naive algorithm. The boundary of image {\bfseries in.pgm} is extended outside its support. \item {\bfseries mode} = 2 : convolution using the FFT. The image {\bfseries in.pgm} is considered as null out of its support. \item {\bfseries mode} = 3 : convolution using the FFT. The boundary of image {\bfseries in.pgm} is extended outside its support.\end{DoxyItemize}
{\bfseries Types supported:} byte 2d, int32\_\-t 2d, float 2d

{\bfseries Category:} signal

\begin{DoxyWarning}{Warning}
Naive convolution algorithm is in O($|$in$|$$\ast$$|$kernel$|$). For large kernels, use FFT version which is in O(n log n) where n = max($|$in$|$,$|$kernel$|$)
\end{DoxyWarning}
\begin{DoxyAuthor}{Author}
Michel Couprie 
\end{DoxyAuthor}
