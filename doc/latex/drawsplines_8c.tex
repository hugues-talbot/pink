\section{drawsplines.c File Reference}
\label{drawsplines_8c}\index{drawsplines.c@{drawsplines.c}}
draw spline segments which are specified by a text file  




\label{_details}
\subsection{Detailed Description}
draw spline segments which are specified by a text file 

{\bf Usage:} drawsplines in.pgm splines.txt [len] out.pgm

{\bf Description:} Draws splines which are specified by control points in a text file. The parameter {\bf in.pgm} gives an image into which the splines are to be drawn. The file format for {\bf splines.txt} is the following for 2D:

The file {\bf splines.txt} contains a list of splines under the format:\par
 d nb\_\-splines\par
 nb\_\-points\_\-spline\_\-1 x11 y11 x12 y12 ...\par
 nb\_\-points\_\-spline\_\-2 x21 y21 x22 y22 ...\par
 nb\_\-points\_\-spline\_\-3 x31 y31 x32 y32 ...\par
 ...\par
 or, in 3D:\par
 D nb\_\-splines\par
 nb\_\-points\_\-spline\_\-1 x11 y11 z11 x12 y12 z12 ...\par
 nb\_\-points\_\-spline\_\-2 x21 y21 z21 x22 y22 z22 ...\par
 nb\_\-points\_\-spline\_\-3 x31 y31 z31 x32 y32 z32 ...\par
 ...\par


If parameter {\bf len} is given and non-zero, the splines are extended on both sides by straight line segments of length {\bf len}.

{\bf Types supported:} byte 2D, byte 3D

{\bf Category:} draw geo

\begin{Desc}
\item[Author:]Michel Couprie \end{Desc}
