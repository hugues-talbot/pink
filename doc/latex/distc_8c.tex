\section{distc.c File Reference}
\label{distc_8c}\index{distc.c@{distc.c}}
distance transform (internal) 



\subsection{Detailed Description}
distance transform (internal) 

{\bf Usage:} distc in.pgm mode out.pgm

{\bf Description:} Distance to the complementary of the object X defined by the binary image {\bf in.pgm} . The result function DX(x) is defined by: DX(x) = min \{d(x,y), y not in X\}.

The distance d used depends on the parameter {\bf mode} : \begin{itemize}
\item 0: approximate euclidean distance (truncated) \item 1: approximate quadratic euclidean distance \item 2: chamfer distance \item 3: exact quadratic euclidean distance \item 4: 4-distance in 2d \item 5: exact euclidean distance (float) \item 8: 8-distance in 2d \item 6: 6-distance in 3d \item 18: 18-distance in 3d \item 26: 26-distance in 3d \item 40: 4-distance in 2d (byte coded ouput) \item 80: 8-distance in 2d (byte coded ouput) \item 60: 6-distance in 3d (byte coded ouput) \item 180: 18-distance in 3d (byte coded ouput) \item 260: 26-distance in 3d (byte coded ouput)\end{itemize}
The output {\bf out.pgm} is of type int32\_\-t for modes $<$ 40, of type byte for other modes.

{\bf Types supported:} byte 2d, byte 3d

{\bf Category:} morpho

\begin{Desc}
\item[Author:]Michel Couprie, Xavier Daragon \end{Desc}
