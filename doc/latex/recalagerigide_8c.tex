\section{recalagerigide.c File Reference}
\label{recalagerigide_8c}\index{recalagerigide.c@{recalagerigide.c}}
rigid registration of two closed contours 



\subsection{Detailed Description}
rigid registration of two closed contours 

{\bf Usage:} recalagerigide in1 in2 out

{\bf Description:}

Rigid registration of two closed contours. Let X and Y be two point sets, given respectively by {\bf in1} and {\bf in2}. This procedure identifies the parameters of a rigid deformation R such that the \char`\"{}distance\char`\"{} between R(X) and Y is a local minimum. The distance D(Z,Y) between two sets Z and Y is defined by:

D(Z,Y) = sum \{ d2(z,Y) ; for all z in Z \}

d2(z,Y) = min \{ d2(z,y) ; for all y in Y \}

d2(z,y) = (z - y)$^\wedge$2 ; i.e., the square or the Euclidean distance between z and y.

The rigid deformation R is defined as the composition (in this order) of scalings, rotations and translations.

In 2d, the parameters are: \begin{itemize}
\item sx : parameter for the scaling in direction x \item sy : parameter for the scaling in direction y \item theta : angle (in radians) of the rotation around the barycenter of X \item tx : parameter for the translation in direction x \item ty : parameter for the translation in direction y\end{itemize}
In 3d, the parameters are: \begin{itemize}
\item sx : parameter for the scaling in direction x \item sy : parameter for the scaling in direction y \item sz : parameter for the scaling in direction z \item theta : angle (in radians) of the rotation around the parallel to the the z axis passing by the barycenter of X \item phi : angle (in radians) of the rotation around the parallel to the the y axis passing by the barycenter of X \item tx : parameter for the translation in direction x \item ty : parameter for the translation in direction y \item tz : parameter for the translation in direction z\end{itemize}
{\bf Types supported:} byte 2d, byte 3d

{\bf Category:} geo

\begin{Desc}
\item[Author:]Michel Couprie \end{Desc}
