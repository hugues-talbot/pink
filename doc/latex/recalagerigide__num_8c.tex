\section{recalagerigide\_\-num.c File Reference}
\label{recalagerigide__num_8c}\index{recalagerigide_num.c@{recalagerigide\_\-num.c}}
rigid registration of two grayscale images 



\subsection{Detailed Description}
rigid registration of two grayscale images 

{\bf Usage:} recalagerigide\_\-num in1 in2 xmin ymin xmax ymax seuil [init] out

{\bf Description:}

Rigid registration of two grayscale images. Let X and Y be two images, given respectively by {\bf in1} and {\bf in2}. This procedure identifies the parameters of a rigid deformation R such that the \char`\"{}distance\char`\"{} between R(X) and Y is a local minimum. The distance D(Z,Y) between two images Z and Y is defined by:

D(Z,Y) = sum \{ (Z[i] - Y[i])$^\wedge$2 ; for all i in support(Y) \}

where support(Y) is a rectangular zone specified by arguments {\bf xmin ymin xmax ymax}.

The rigid deformation R is defined as the composition (in this order) of a zoom, a rotation, and a translation.

In 2d, the parameters are: \begin{itemize}
\item hx : parameter for the scaling in direction x \item hy : parameter for the scaling in direction y \item theta : angle (in degrees) of the rotation around (0,0) \item tx : parameter for the translation in direction x \item ty : parameter for the translation in direction y\end{itemize}
The output parameter {\bf out} is the name of a text file in which these values will be written (type \char`\"{}e\char`\"{} list format, see doc/formats.txt).

The parameter {\bf seuil} makes it possible to eliminate outliers: points i such that (Z[i] - Y[i])$^\wedge$2 $>$ seuil$^\wedge$2 are not taken into account. A value 255 for this parameter means no outlier elimination.

The optional parameter {\bf init} makes it possible to give, in the same format as the output, an initial deformation which is \char`\"{}close\char`\"{} to the expected one. The default initialisation is the identity (parameters 0, 1, 1, 0, 0).

{\bf Types supported:} byte 2d

{\bf Category:} geo

\begin{Desc}
\item[Author:]Michel Couprie \end{Desc}
