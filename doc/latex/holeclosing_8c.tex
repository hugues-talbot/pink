\section{holeclosing.c File Reference}
\label{holeclosing_8c}\index{holeclosing.c@{holeclosing.c}}
3d hole closing  




\label{_details}
\subsection{Detailed Description}
3d hole closing 

{\bf Usage:} holeclosing in connex holesize out

{\bf Description:} Hole closing in 3d binary images. The parameter {\bf connex} gives the connectivity used for the object; possible choices are 6 and 26. Holes which have a \char`\"{}size\char`\"{} greater than {\bf holesize} are let open (where -1 is used as a symbol for infinity).

Let X be the set of points of the binary image {\bf in}, let Y be a full enclosing box. The algorithm is the following:



\begin{footnotesize}\begin{verbatim}
Repeat until stability:
    Select a point p of Y \ X such that Tb(p,Y) = 1
        or such that Tb(p,Y) = 2 and d(p,X) > holesize
        which is at the greatest distance from X
    Y := Y \ {p}
Result: Y
\end{verbatim}
\end{footnotesize}


Reference: Z. Aktouf, G. Bertrand, L.Perroton: \char`\"{}A three-dimensional holes closing algorithm\char`\"{}, {\em Pattern Recognition Letters\/}, No.23, pp.523-531, 2002.

{\bf Types supported:} byte 3d

{\bf Category:} topobin

\begin{Desc}
\item[Author:]Michel Couprie \end{Desc}
