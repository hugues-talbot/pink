\section{Avertissement}\label{index_warning}
Copyright ESIEE (2009)

{\tt m.couprie@esiee.fr}

This software is an image processing library whose purpose is to be used primarily for research and teaching.

This software is governed by the CeCILL license under French law and abiding by the rules of distribution of free software. You can use, modify and/ or redistribute the software under the terms of the CeCILL license as circulated by CEA, CNRS and INRIA at the following URL \char`\"{}http://www.cecill.info\char`\"{}.

As a counterpart to the access to the source code and rights to copy, modify and redistribute granted by the license, users are provided only with a limited warranty and the software's author, the holder of the economic rights, and the successive licensors have only limited liability.

In this respect, the user's attention is drawn to the risks associated with loading, using, modifying and/or developing or reproducing the software by the user in light of its specific status of free software, that may mean that it is complicated to manipulate, and that also therefore means that it is reserved for developers and experienced professionals having in-depth computer knowledge. Users are therefore encouraged to load and test the software's suitability as regards their requirements in conditions enabling the security of their systems and/or data to be ensured and, more generally, to use and operate it in the same conditions as regards security.

The fact that you are presently reading this means that you have had knowledge of the CeCILL license and that you accept its terms.\section{Contributeurs}\label{index_contributors}
Michel Couprie\par
 Laurent Najman: localextrema, saliency\par
 Hugues Talbot: fmm \par
 Jean Cousty: redt 3d (reverse euclidean distance transform - algo de D. Coeurjolly), watershedthin, op�rateurs sur les graphes d'ar�tes (GA), for�ts de poids min (MSF), waterfall, recalagerigide\_\-translateplane\par
 Xavier Daragon: dist, distc (distance euclidienne quadratique 3D)\par
 Andr� Vital Saude: radialopening, divers scripts tcl, hma\par
 Nicolas Combaret: toposhrinkgray, ptselectgray\par
 John Chaussard: lballincl, cropondisk, shrinkondisk\par
 Christophe Doublier: zoomint\par
 Hildegard Koehler: lintophat\par
 C�dric All�ne: gettree, histolisse, labeltree, nbcomp, pgm2vtk, seuilauto\par
 Gu Jun: maxdiameter\par
 S�bastien Couprie: mcsplines.c\par
 Rita Zrour: medialaxis (axe m�dian euclidien exact - algo de R�my-Thiel), dist, distc (distance euclidienne quadratique exacte - algo de Saito-Toriwaki)\par
 Laurent Mercier: gestion d'un masque dans delaunay\par


CODE UNDER FREE LICENCE INCLUDED

David Coeurjolly: lvoronoilabelling.c\par
 Dario Bressanini: mcpowell.c\par
 Andrew W. Fitzgibbon: lbresen.c\par
 Lilian Buzer: lbdigitalline.cxx\section{Installation}\label{index_install}
\subsection{Linux}\label{index_linux}
Pour installer la biblioth�que sous Linux, tapez: 

\footnotesize\begin{verbatim}
tar zxvf pink.tgz
mv Pinktmp Pink
cd Pink
makelin
\end{verbatim}
\normalsize


Les ex�cutables se trouvent dans Pink/linux/bin, les scripts dans Pink/scripts et dans Pink/tcl Mettez � jour votre variable \$path en cons�quence. Par exemple (en csh):



\footnotesize\begin{verbatim}
setenv PINK ~coupriem/Pink
set path=( $path $PINK/linux/bin $PINK/scripts $PINK/tcl )
\end{verbatim}
\normalsize
\subsection{Windows}\label{index_windows}
Installez tout d'abord {\tt MinGW} (Minimalist GNU For Windows), ainsi que le programme \char`\"{}make\char`\"{} de la distribution MinGW (installation s�par�e).

D�compressez l'archive pink.tgz (par exemple avec WinZip ou WinRar).

Modifiez selon votre configuration, les fichiers \char`\"{}mgw.make\char`\"{} et \char`\"{}makemgw.bat\char`\"{}.

Ouvrez une fen�tre DOS et mettez-vous dans le r�pertoire principal de Pink.

tapez: makemgw

Les ex�cutables se trouvent dans Pink/bin, mettez � jour votre variable PATH en cons�quence. Par exemple :



\footnotesize\begin{verbatim}
set PATH=c:\Pink\bin;%PATH%
\end{verbatim}
\normalsize


Erreurs de compilation possibles et solutions:

message d'erreur contenant \char`\"{}M\_\-PI\char`\"{} cause: version MinGW obsolete solution rapide: recopier le contenu du fichier Pink/tools/pinkconst.h dans le fichier MinGW/include/math.h

message d'erreur contenant \char`\"{}uint8\_\-t\char`\"{}, \char`\"{}int8\_\-t\char`\"{}, \char`\"{}uint16\_\-t\char`\"{}, etc cause: version MinGW obsolete solution rapide: recopier le contenu du fichier Pink/tools/pinktypes.h dans le fichier MinGW/include/math.h

message d'erreur contenant \char`\"{}chrono\char`\"{} cause: flag de mise au point oubli� solution rapide: chercher le \char`\"{}\#define CHRONO\char`\"{} dans le fichier source et le retirer\subsection{Logiciels associ�s}\label{index_associes}
Pour une utilisation optimale, les logiciels suivants doivent �tre install�s:\par
\par
 {\tt Doxygen}\par
 {\tt ActiveTcl 8.3}\par
 {\tt VTK}\par
 {\tt MPlayer}\par
 {\tt Gnuplot}\par




 Michel Couprie - Professeur - ESIEE Paris Laboratoire d'Informatique Gaspard-Monge, Universit� Paris-Est ESIEE 2, Bd Blaise Pascal - B.P. 99 93162 Noisy-Le-Grand CEDEX mail: {\tt m.couprie@esiee.fr} url: {\tt http://www.esiee.fr/$\sim$coupriem} tel: 01 45 92 66 88 fax: 01 45 92 66 99 