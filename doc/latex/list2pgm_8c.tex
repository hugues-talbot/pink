\section{list2pgm.c File Reference}
\label{list2pgm_8c}\index{list2pgm.c@{list2pgm.c}}
converts from point list representation to pgm  




\label{_details}
\subsection{Detailed Description}
converts from point list representation to pgm 

{\bf Usage:} list2pgm in.list $<$in.pgm$|$rs cs ds$>$ [scale] out.pgm

{\bf Description:}

Reads the file {\bf in.list}. This file must have one of the following formats: \small\begin{alltt}  
  e <n>       s <n>         b <n>         n <n>            B <n>            N <n>    
  x1          x1 v1         x1 y1         x1 y1 v1         x1 y1 z1         x1 y1 z1 v1
  x2    or    x2 v2   or    x2 y2   or    x2 y2 v2   or    x2 y2 z2   or    z2 x2 y2 v2
  ...         ...           ...           ...              ...              ...
  xn          xn vn         xn yn         xn yn vn         xn yn zn         zn xn yn vn
\end{alltt}
\normalsize 
 The formats {\bf e}, {\bf s}, {\bf b}, {\bf n}, {\bf B}, and {\bf N}, correspond respectively to binary 1D, graylevel 1D, binary 2D, graylevel 2D, binary 3D and graylevel 3D images. If a file name {\bf in.pgm} is given, then the points read in {\bf in.list} are inserted (if possible) into the image read in {\bf in.pgm}. Else, they are inserted in a new image, the dimensions of which are {\bf rs}, {\bf cs}, and {\bf d}.

The optional parameter {\bf scale} allows to scale the coordinates.

{\bf Types supported:} list 1D, 2D, 3D

{\bf Category:} convert

\begin{Desc}
\item[Author:]Michel Couprie \end{Desc}
