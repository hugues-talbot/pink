\section{herosball.c File Reference}
\label{herosball_8c}\index{herosball.c@{herosball.c}}
topologically controlled erosion 



\subsection{Detailed Description}
topologically controlled erosion 

{\bf Usage:} herosball in.pgm radius dist connex out.pgm

{\bf Description:} Performs a topologically controlled erosion, that is, a homotopic thinning constrained by the erosion of the input object.

The parameter {\bf radius} gives the radius of the ball which is used as structuring element for the erosion.

The parameter {\bf dist} is a numerical code indicating the distance which is used to compute the erosion. The possible choices are: \begin{itemize}
\item 0: approximate euclidean distance \item 1: approximate quadratic euclidean distance \item 2: chamfer distance \item 3: exact quadratic euclidean distance \item 4: 4-distance in 2d \item 8: 8-distance in 2d \item 6: 6-distance in 3d \item 18: 18-distance in 3d \item 26: 26-distance in 3d\end{itemize}
The parameter {\bf connex} indicates the connectivity of the binary object. Possible choices are 4, 8 in 2d and 6, 26 in 3d.

Let X be the set corresponding to the input image {\bf in.pgm}, and let Xbar denote its complementary set. The algorithm is the following:



\footnotesize\begin{verbatim}I = erosball(X, dist, r)
Repeat:
    Select a point x in X \ I such that dist(x,Xbar) is minimal
    If x is simple for X then
        X = X \ {x}
	I = I union {x}
Until X == I
Result: X
\end{verbatim}
\normalsize


{\bf Types supported:} byte 2d, byte 3d

{\bf Category:} topobin

\begin{Desc}
\item[Author:]Michel Couprie \end{Desc}
