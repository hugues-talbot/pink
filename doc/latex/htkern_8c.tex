\section{htkern.c File Reference}
\label{htkern_8c}\index{htkern.c@{htkern.c}}
grayscale ultimate homotopic thinning  




\label{_details}
\subsection{Detailed Description}
grayscale ultimate homotopic thinning 

{\bf Usage:} htkern in.pgm $<$imcond.pgm$|$null$>$ connex out.pgm

{\bf Description:} Grayscale ultimate homotopic thinning (refs. [BEC97, CBB01]). The parameter {\bf connex} gives the connectivity used for the minima; possible choices are 4, 8 in 2D and 6, 26 in 3D. Let F be the function corresponding to the input image {\bf in.pgm}. Let G be the function corresponding to the input image {\bf imcond.pgm}, or the null function if the keyword {\bf null} is used. The algorithm is the following:



\begin{footnotesize}\begin{verbatim}
Repeat until stability:
    select p destructible for F such that F(p) is minimal
    F(p) = max{ delta-(p,F), G(p) }    
Result: F
\end{verbatim}
\end{footnotesize}


References:\par
 [BEC97] G. Bertrand, J. C. Everat and M. Couprie: {\tt \char`\"{}Image segmentation through operators based upon topology\char`\"{}}, {\em  Journal of Electronic Imaging\/}, Vol.~6, No.~4, pp.~395-405, 1997.\par
 [CBB01] M. Couprie, F.N. Bezerra, Gilles Bertrand: {\tt \char`\"{}Topological operators for grayscale image processing\char`\"{}}, {\em  Journal of Electronic Imaging\/}, Vol.~10, No.~4, pp.~1003-1015, 2001.

{\bf Types supported:} byte 2D, byte 3D.

{\bf Category:} topogray

\begin{Desc}
\item[Author:]Michel Couprie\end{Desc}
{\bf Example:}

htkern ur1 null 4 ur1\_\-htkern\par
 minima ur1\_\-htkern 4 ur1\_\-htkern\_\-m

\begin{TabularC}{3}
\hline
 & &  \\\cline{1-3}
ur1 &ur1\_\-htkern &ur1\_\-htkern\_\-m  \\\cline{1-3}
\end{TabularC}
