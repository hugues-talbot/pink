\section{drawspline.c File Reference}
\label{drawspline_8c}\index{drawspline.c@{drawspline.c}}
draw a spline which is specified by its control points in a text file 



\subsection{Detailed Description}
draw a spline which is specified by its control points in a text file 

{\bf Usage:} drawspline in.pgm spline.txt [len] out.pgm

{\bf Description:} Draws a spline which is specified by its control points in a text file. The parameter {\bf in.pgm} gives an image into which the spline is to be drawn. The file format for {\bf spline.txt} is the following for 2D:

s n+1 (where n+1 denotes the number of control points)\par
 x1 y1\par
 ...\par
 xn+1 yn+1\par
 C0X1 C0Y1 C1X1 C1Y1 C2X1 C2Y1 C3X1 C3Y1\par
 ...\par
 C0Xn C0Yn C1Xn C1Yn C2Xn C2Yn C3Xn C3Yn\par


and in the 3D case:

S n+1 (where n+1 denotes the number of control points)\par
 x1 y1 z1\par
 ...\par
 xn+1 yn+1 zn+1\par
 C0X1 C0Y1 C0Z1 C1X1 C1Y1 C1Z1 C2X1 C2Y1 C2Z1 C3X1 C3Y1 C3Z1\par
 ...\par
 C0Xn C0Yn C0Zn C1Xn C1Yn C1Zn C2Xn C2Yn C2Zn C3Xn C3Yn C3Zn\par


If parameter {\bf len} is given and non-zero, the spline is extended on both sides by straight line segments of length {\bf len}.

{\bf Types supported:} byte 2D, byte 3D

{\bf Category:} draw geo

\begin{Desc}
\item[Author:]Michel Couprie \end{Desc}
