\section{pgm2closedcurve.c File Reference}
\label{pgm2closedcurve_8c}\index{pgm2closedcurve.c@{pgm2closedcurve.c}}
extracts a curve from a binary image 



\subsection{Detailed Description}
extracts a curve from a binary image 

{\bf Usage:} pgm2closedcurve image.pgm connex [x y [z]] out.curve

{\bf Description:} Extracts a closed curve from a binary image, that is, a curve with the same starting and ending point. The parameter {\bf connex} is the connexity of the curve. It may be equal to 4 or 8 in 2D, and to 6, 18 or 26 in 3D. If given, the point {\bf (x, y, z)} (2D) or {\bf (x, y, z)} (3D) is taken as the starting point of the curve, and must be a curve point. The output is the text file {\bf out.curve}, with the following format:\par
 c nbpoints\par
 x1 y1\par
 x2 y2\par
 ...\par
 or (3D): C nbpoints\par
 x1 y1 z1\par
 x2 y2 z2\par
 ...\par


The points of the curve may also be valued. This is must be indicated by a value of 40, 80, 60, 180 or 260 for the parameter {\bf connex}, instead of 4, 8, 6, 18 or 26 respectively. In this case, the output is the text file {\bf out.curve}, with the following format:\par
 cv nbpoints\par
 x1 y1 v1\par
 x2 y2 v2\par
 ...\par
 or (3D): CV nbpoints\par
 x1 y1 z1 v1\par
 x2 y2 z2 v2\par
 ...\par


{\bf Types supported:} byte 2D, byte 3D

{\bf Category:} convert geo

\begin{Desc}
\item[Author:]Michel Couprie \end{Desc}
