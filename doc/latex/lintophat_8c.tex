\section{lintophat.c File Reference}
\label{lintophat_8c}\index{lintophat.c@{lintophat.c}}


max of morphological black top hats by linear structuring elements  




\subsection{Detailed Description}
max of morphological black top hats by linear structuring elements {\bfseries Usage:} lintophat in.pgm length out.pgm

{\bfseries Description:} Max of morphological black top hats by linear structuring elements, in all possible discrete directions. Let F be the original image and E be a structuring element, the black top hat of F by E is defined by F -\/ closing(F, E). The closing deletes dark structures that do not match the structuring element, thus the black top hat detects those dark structures. For a linear structuring element, the detected structures are those which are orthogonal to the se. The length if the linear structuring elements is given by {\bfseries length}.

{\bfseries Types supported:} byte 2d

{\bfseries Category:} morpho

\begin{DoxyAuthor}{Author}
Hildegard Koehler 2003 
\end{DoxyAuthor}
