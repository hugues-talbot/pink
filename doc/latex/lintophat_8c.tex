\section{lintophat.c File Reference}
\label{lintophat_8c}\index{lintophat.c@{lintophat.c}}
max of morphological black top hats by linear structuring elements  




\label{_details}
\subsection{Detailed Description}
max of morphological black top hats by linear structuring elements 

{\bf Usage:} lintophat in.pgm length out.pgm

{\bf Description:} Max of morphological black top hats by linear structuring elements, in all possible discrete directions. Let F be the original image and E be a structuring element, the black top hat of F by E is defined by F - closing(F, E). The closing deletes dark structures that do not match the structuring element, thus the black top hat detects those dark structures. For a linear structuring element, the detected structures are those which are orthogonal to the se. The length if the linear structuring elements is given by {\bf length}.

{\bf Types supported:} byte 2d

{\bf Category:} morpho

\begin{Desc}
\item[Author:]Hildegard Koehler 2003 \end{Desc}
