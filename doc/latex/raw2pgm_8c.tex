\section{raw2pgm.c File Reference}
\label{raw2pgm_8c}\index{raw2pgm.c@{raw2pgm.c}}
converts from raw format into pgm format  




\label{_details}
\subsection{Detailed Description}
converts from raw format into pgm format 

{\bf Usage:} in.raw rs cs ds headersize datatype littleendian [xdim ydim zdim] out.pgm

{\bf Description:} Converts from raw format into pgm format.

Parameters: \begin{itemize}
\item {\bf in.pgm} : source file in raw format \item {\bf rs} (int32\_\-t): row size (number of voxels in a row) \item {\bf cs} (int32\_\-t): column size (number of voxels in a column) \item {\bf ds} (int32\_\-t): number of planes \item {\bf headersize} (int32\_\-t): size of the header in bytes (information in the header will be ignored) \item {\bf datatype} (int32\_\-t): (1 for byte, 2 for short int, 4 for long int, 5 for float) \item {\bf littleendian} (int32\_\-t) 1: littleendian, 0: bigendian. Usual choice is 0. \item {\bf xdim} (float, optional) : gap (in the real world) between two adjacent voxels in a row. \item {\bf ydim} (float, optional) : gap (in the real world) between two adjacent voxels in a column. \item {\bf zdim} (float, optional) : gap (in the real world) between two adjacent planes.\end{itemize}
{\bf Types supported:} byte 3D, int16\_\-t 3D, int32\_\-t 3D

\begin{Desc}
\item[Warning:]Signed integers and floating numbers are not supported.\end{Desc}
{\bf Category:} convert

\begin{Desc}
\item[Author:]Michel Couprie \end{Desc}
