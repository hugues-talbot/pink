\section{skelpar3d.c File Reference}
\label{skelpar3d_8c}\index{skelpar3d.c@{skelpar3d.c}}
parallel 3D binary skeleton  




\label{_details}
\subsection{Detailed Description}
parallel 3D binary skeleton 

{\bf Usage:} skelpar3d in.pgm algorithm nsteps [inhibit] out.pgm

{\bf Description:} Parallel 3D binary thinning or skeleton. The parameter {\bf nsteps} gives, if positive, the number of parallel thinning steps to be processed. If the value given for {\bf nsteps} equals -1, the thinning is continued until stability.

The parameter {\bf algorithm} is a numerical code indicating which method will be used for the thinning. The possible choices are: \begin{itemize}
\item 0: ultimate, without constraint (MK3a) \item 1: curvilinear, based on 1D isthmus (CK3a) \item 2: medial axis preservation (AK3) - parameter inhibit represents the minimal radius of medial axis balls which are considered \item 3: ultimate (MK3) - if nsteps = -2, returns the topological distance \item 4: curvilinear based on ends (EK3) \item 5: curvilinear based on ends, with end reconstruction (CK3b) \item 6: topological axis (not homotopic) \item 7: curvilinear, based on residual points and 2D isthmus (CK3) \item 8: ultimate, asymetric (AMK3) \item 9: curvilinear, asymetric, based on thin 1D isthmus (ACK3a) \item 10: curvilinear, asymetric, based on 3D and 2D residuals (ACK3) \item 11: surfacic, based on residual points (RK3) \item 12: surfacic, based on 2D isthmuses (SK3) \item 13: ultimate, directional, (DK3) \item 14: surfacic, directional, based on residual points (DRK3) \item 15: surfacic, directional, based on 2D isthmuses (DSK3)\end{itemize}
In modes other than 2, if the parameter {\bf inhibit} is given and is a binary image name, then the points of this image will be left unchanged.

The following codes give access to auxiliary functions, for isthmus detection. Parameter {\bf nsteps} will not be used in this case (any value can be given).

\begin{itemize}
\item 100: 1D isthmus points\end{itemize}
{\bf Warning:} The object must not have any point on the frame of the image.

{\bf Types supported:} byte 3d

{\bf Category:} topobin

\begin{Desc}
\item[Author:]Michel Couprie \end{Desc}
