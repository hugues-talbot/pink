\section{pgm2vskel.c File Reference}
\label{pgm2vskel_8c}\index{pgm2vskel.c@{pgm2vskel.c}}
decomposition of a valuated curvilinear skeleton into isolated points, end points, curves and junctions  




\label{_details}
\subsection{Detailed Description}
decomposition of a valuated curvilinear skeleton into isolated points, end points, curves and junctions 

{\bf Usage:} pgm2vskel in.pgm val.pgm connex out.skel

{\bf Description:} The skeleton found in {\bf in.pgm} is decomposed into isolated points, end points, curves and junctions ; and its description is stored in {\bf out.skel} . Each point of the skeleton is valuated by the corresponding value found in {\bf val.pgm} . The parameter {\bf connex} sets the adjacency relation used for the object (4, 8 (2d) or 6, 18, 26 (3d)).

\begin{Desc}
\item[Warning:]No verification is done to check that the input image {\bf in.pgm} contains indeed a curvilinear skeleton. In the contrary case, the result would be meaningless.

IMPORTANT LIMITATION: different junctions in the original image must not be in direct contact with each other (i.e., connected) otherwise they will be considered as a single junction. To prevent this to occur, one can increase image resolution.\end{Desc}
{\bf Types supported:} byte 2d, byte 3d

{\bf Category:} topobin

\begin{Desc}
\item[Author:]Michel Couprie 2004 \end{Desc}
