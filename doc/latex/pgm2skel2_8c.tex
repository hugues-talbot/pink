\section{pgm2skel2.c File Reference}
\label{pgm2skel2_8c}\index{pgm2skel2.c@{pgm2skel2.c}}
decomposition of a curvilinear skeleton into isolated points, end points, curves and junctions  




\label{_details}
\subsection{Detailed Description}
decomposition of a curvilinear skeleton into isolated points, end points, curves and junctions 

{\bf Usage:} pgm2skel in.pgm junc.pgm connex out.skel

{\bf Description:} The skeleton found in {\bf in.pgm} is decomposed into isolated points, end points, curves and junctions ; and its description is stored in {\bf out.skel} . The parameter {\bf connex} sets the adjacency relation used for the object (4, 8 (2d) or 6, 18, 26 (3d)).

The image given as parameter{\bf junc.pgm} contains curve points that will artificially considered as junction points.

\begin{Desc}
\item[Warning:]Points at the border of the image will be ignored.

IMPORTANT LIMITATION: different junctions in the original image must not be in direct contact with each other (i.e., connected) otherwise they will be considered as a single junction. To prevent this to occur, one can increase image resolution.\end{Desc}
{\bf Types supported:} byte 2d, byte 3d

{\bf Category:} topobin

\begin{Desc}
\item[Author:]Michel Couprie 2009 \end{Desc}
