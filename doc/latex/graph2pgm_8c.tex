\section{graph2pgm.c File Reference}
\label{graph2pgm_8c}\index{graph2pgm.c@{graph2pgm.c}}
converts from graph representation to pgm 



\subsection{Detailed Description}
converts from graph representation to pgm 

{\bf Usage:} graph2pgm in.graph $<$in.pgm$|$rs cs ds$>$ out.pgm

{\bf Description:}

Reads the file {\bf in.graph}. Each vertex of this graph must have integer coordinates, and represents a pixel/voxel of a 2D/3D image. If a file name {\bf in.pgm} is given, then the points read in {\bf in.graph} are inserted (if possible) into the image read in {\bf in.pgm}. Else, they are inserted in a new image, the dimensions of which are {\bf rs}, {\bf cs}, and {\bf ds}.

{\bf Types supported:} graph

{\bf Category:} convert

\begin{Desc}
\item[Author:]Michel Couprie \end{Desc}
