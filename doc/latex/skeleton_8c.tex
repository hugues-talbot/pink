\section{skeleton.c File Reference}
\label{skeleton_8c}\index{skeleton.c@{skeleton.c}}


ultimate binary skeleton guided by a priority image (see [BC07])  




\subsection{Detailed Description}
ultimate binary skeleton guided by a priority image (see [BC07]) {\bfseries Usage:} skeleton in.pgm prio connex [inhibit] out.pgm

{\bfseries Description:} Ultimate binary skeleton guided by a priority image. The lowest values of the priority image correspond to the highest priority.

The parameter {\bfseries prio} is either an image (byte, int32\_\-t, float or double), or a numerical code indicating that a distance map will be used as a priority image; the possible choices are: \begin{DoxyItemize}
\item 0: approximate euclidean distance \item 1: approximate quadratic euclidean distance \item 2: chamfer distance \item 3: exact quadratic euclidean distance \item 4: 4-\/distance in 2d \item 8: 8-\/distance in 2d \item 6: 6-\/distance in 3d \item 18: 18-\/distance in 3d \item 26: 26-\/distance in 3d\end{DoxyItemize}
The parameter {\bfseries connex} indicates the connectivity of the binary object. Possible choices are 4, 8 in 2d and 6, 26 in 3d.

If the parameter {\bfseries inhibit} is given and is an integer different from -\/1, then the points which correspond to this priority value will be left unchanged. If the parameter {\bfseries inhibit} is given and is a binary image name, then the points of this image will be left unchanged.

Let X be the set corresponding to the input image {\bfseries in.pgm}. Let P be the function corresponding to the priority image. Let I be the set corresponding to the inhibit image, if given, or the empty set otherwise. The algorithm is the following:

\begin{DoxyVerb}
Repeat until stability
    Select a point x in X \ I such that P[x] is minimal
    If x is simple for X then
        X = X \ {x}
Result: X
\end{DoxyVerb}


Reference:\par
 [BC07] G. Bertrand and M. Couprie: {\tt \char`\"{}Transformations topologiques discretes\char`\"{}}, in {\itshape G\'{e}om\'{e}trie discr\`{e}te et images num\'{e}riques\/}, D. Coeurjolly and A. Montanvert and J.M. Chassery, pp.~187-\/209, Herm\`{e}s, 2007.\par
 {\bfseries Types supported:} byte 2d, byte 3d

{\bfseries Category:} topobin

\begin{DoxyAuthor}{Author}
Michel Couprie
\end{DoxyAuthor}
{\bfseries Example:}

skeleton circuit1 8 8 circuit1\_\-skeleton

\begin{TabularC}{2}
\hline
 &  \\\cline{1-2}
circuit1 &circuit1\_\-skeleton  \\\cline{1-2}
\end{TabularC}
