\chapter{Introduction}\label{chap:introduction}

\Pink is an image processing library developed at \href {http://www.esiee.fr}{ESIEE
  Paris} for research and teaching purposes. It
contains implementations of over 200 algorithms for image segmentation and
filtering. Most of the operators come from mathematical morphology, but it
contains operators from different fields. Pink is free software licensed under
the CeCILL license.

We are interested in the continuous development of Pink. It has already been
proven useful in many applications and we are constantly looking for new ones.

This manual aims at referencing most functions of \Pink, hopefully in a didactic manner.

\section*{History}
Pink is Not Khoros

\section*{Related software}
These days, image processing is a technology and something such as \Pink is best
used as a component among others. For optimal use, the following packages should
be installed:

\begin{itemize}
\item \href {http://www.sf.net/projects/imview}{imview}
\item Python (version 2.7 preferred) ; with the following packages:
  \begin{itemize}
  \item numpy
  \item scipy
  \item matplotlib
  \item python-vtk
  \item python-image (PIL)
 \item python-image-tk
  \end{itemize}
\item Doxygen
\item ActiveTcl 8.3
\item VTK
\item MPlayer
\item Gnuplot
\end{itemize}

Other software may prove useful: 

\begin{itemize}
\item OpenCV
\item ITK
\end{itemize}

\section*{Contributors}
\Pink is the result of many thousands of hours of work, and includes
contributions from this (non-exhaustive) list of people

Code under the main CeCILL license:
\begin{itemize}
\item Michel Couprie : main author, initial design
\item László Marak (ujoimro) : library, port to Python, continuous maximum flows, Total-Variation denoising, Python front-end, native Microsoft Windows port.
\item Laurent Najman : localextrema, saliency
\item Hugues Talbot : fmm, fast morphological operators, region growing; this documentation.
\item Jean Cousty : redt 3d (reverse euclidean distance transform - algo de D. Coeurjolly), watershedthin, opérateurs sur les graphes d'arêtes (GA), minimum cost forest (MSF), waterfall, recalagerigide translateplane
\item Xavier Daragon: dist, distc (Quadratic Euclidian Distance in 3D)
\item André Vital Saude: radialopening, divers scripts tcl, hma
\item Nicolas Combaret: toposhrinkgray, ptselectgray
\item John Chaussard: lballincl, cropondisk, shrinkondisk
\item Christophe Doublier: zoomint
\item Hildegard Koehler: lintophat
\item Cédric Allène: gettree, histolisse, labeltree, nbcomp, pgm2vtk, seuilauto
\item Gu Jun: maxdiameter
\item Sébastien Couprie: mcsplines.c
\item Rita Zrour: medialaxis (exact Euclidean medial axis - algorithm of Rémy Thiel), dist, distc (Quadratic exact Euclidean distance - algorithm of Saito-Toriwaki, in 2D)
\item Laurent Mercier: gestion d'un masque dans delaunay
\item Benjamin Raynal: parallel 3D thinning
\item Nivando Bezerra: parallel grayscale thinning
\end{itemize}

Code under different free software licenses:
\begin{itemize}
\item David Coeurjolly: lvoronoilabelling.c
\item Dario Bressanini: mcpowell.c
\item Andrew W. Fitzgibbon: lbresen.c
\item Lilian Buzer: lbdigitalline.cxx
\end{itemize}