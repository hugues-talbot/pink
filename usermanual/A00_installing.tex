%%
%% Installation
%%

\chapter{Compiling and installing \Pink}\label{chap:installing}
While not as difficult as it may seem, obtaining, compiling and installing \Pink
is currently an imperfect process.

We recommend you try to get a binary version from
\url{http://pinkhq.com}. If you cannot find what you want there, then
by all means get the Git version:

\begin{verbatim}
hg clone https://www.pinkhq.com/hg/pink
\end{verbatim}

You may need to work around your local firewall, but this particular
arcane branch of knowledge is outside the scope of this document.

\section{Git branches}
There may be several branches of interest to you. Please read the
mercurial {\tt hg} documentation to get their names and revision
history. Various GUI tools exists that may make this task
easier. Typically the main branch is the one that is known to compile
and work on the platform we do use. However several development
branches are currently active. If you are involved in \Pink
development, you may know which branch is of interest to you. If you
feel adventurous, by all means try the branch that is most
recent. Once you have made your choice, compiling is the next task.


\section{Compiling}
For compiling \Pink, you need the following components

\begin{itemize}
\item CMake version 
\end{itemize}